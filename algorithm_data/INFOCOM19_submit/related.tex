\vspace{-0.1cm}
\section{Related Work}
\vspace{-0.1cm}


% P4 协议无关转发使得有机会任意修改包头
% INT是一种P4的应用 可以采集数据平面状态,采集过程中不打扰控制平面,INT缺点
% barefoot公司的保密应用,及其问题
% 一些人也实现了 但是没有讨论
% 在2015年sigcomm有人提出pingmesh,但是主要采集endhost的实时性能数据
% 大量的应用提出基于底层数据进行ai闭环控制,但是并没有给出如何做的,我们给出了一种具体的实现方法

The proposal of INT-path builds on the giant shoulders of past researches. \cite{bosshart2014p4} invents P4, a protocol-independent packet processing architecture as well as the programming language, which enables network operators to arbitrarily modify packet headers. INT~\cite{kim2015band} is a telemetry application using P4, which can obtain device-internal states without disturbing the controller. However, INT itself is an underlying telemetry primitive for one device or a device chain without defining how to achieve network-wide telemetry, which further requires high-level orchestration. There is a very related piece of work named ``Barefoot Deep Insight'', which is also built upon INT and claims to enable \emph{end-to-end} traffic monitoring~\cite{deepinsight}. However, Barefoot Deep Insight is a proprietary solution without disclosing any technical detail. Some other works such as~\cite{van2017towards} conduct solid telemetry implementation but still do not touch the high-level orchestration. In 2015, Pingmesh~\cite{guo2015pingmesh} conducts a similar network-wide telemetry in one of Microsoft's data centers. However, Pingmesh only measures the end-host performance without diving into network devices. After all, P4 and INT are not invented at that time. As the rise of the AI age, many network AI systems claims to build closed-loop control systems depending on real-time traffic status collection~\cite{ mestres2017knowledge}. However, they talk less on the telemetry detail and our system can be a solid complement for their intelligent systems.