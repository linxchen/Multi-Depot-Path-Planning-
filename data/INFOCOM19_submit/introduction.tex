\section{Introduction}

Today's data center networks have become mega-scale with increasing deployment of diverse cloud services. With the continuous expansion of network size, fine-grained network monitoring becomes the prerequisite for better network reliability and closed-loop traffic control. In traditional network monitoring, management protocols, such as SNMP~\cite{case1990simple}, are widely adopted to constantly poll the router/switch CPU for collecting device-internal states every few seconds or minutes. However, due to the frequent interaction between the control plane and the data plane as well as the limited CPU capability, such monitoring mechanism is coarse-grained and involves a large query latency, which cannot scale well in today's high-density data center networks with drastic traffic dynamics.
 
To ameliorate the scalability issue, In-band Network Telemetry (INT) is proposed by the P4 Language Consortium (P4.org) to achieve fine-grained real-time data plane monitoring~\cite{kim2015band}. INT allows packets to query device-internal states such as queue depth, queuing latency when they pass through the data plane pipeline, without requiring additional intervention by the control plane CPU. Typically, INT relies on a \emph{probe packet} with a variable-length label stack reserved in the packet header. The probe packets are periodically generated from an INT agent and injected into the network, where the probe packets will be queued and forwarded with the ordinary traffic. In each router/switch along the forwarding path, the probe packet will extract device-internal states and push them into the INT label stack. At the last hop, the probe packet containing the end-to-end monitoring data will be sent to the remote controller for further analysis.

INT is essentially an underlying primitive that need the support of special hardware for internal state exposure. With such data extraction interface, network operators can easily obtain the real-time traffic status of a single device or a device chain along the probing path. However, to improve network management, INT further requires a high-level mechanism built upon it to efficiently extract the \emph{network-wide} traffic status. More specifically, as Software-Defined Networking (SDN) is widely deployed, the controller always expects a \emph{global} view (\ie, network-wide visibility) to make the optimal traffic control decisions. Besides, network management automation via machine learning also requires timely feedback from the environment as the fed-in training data~\cite{mestres2017knowledge}. 

In this work, we raise the concept of ``In-band Network-Wide Telemetry'' and propose \emph{INT-path}, a telemetry framework to achieve network-wide traffic monitoring. We tackle the problem using the divide-and-conquer design philosophy by decoupling the solution into a routing \emph{mechanism} and a routing path generation \emph{policy}. Specifically, we embed \emph{source routing} into INT probes to allow specifying the route the probe packet takes through the network. Based on the routing mechanism, we design two INT path planning policies to generate multiple non-overlapped INT paths that cover the entire network. The first is based on \emph{depth-first search} (DFS) which is straightforward but time-efficient. The second is an \emph{Euler trail-based} algorithm that optimally generates non-overlapped INT paths with a \emph{minimum} path number.
 
Based on INT, our approach can ``encode'' the network-wide traffic status into a series of ``bitmap images'', which further allows using advanced techniques, such as pattern recognition, to automate network monitoring and troubleshooting (although pattern recognition has been widely used in image processing, we rarely find their usage in networking area). Such transformation from traffic status to bitmap images will have profound significance because when the network becomes mega-scale, automated approaches are more efficient than traffic analysis purely by human efforts. Coincidentally, there is a very related piece of work named ``Barefoot Deep Insight'', which is also built upon INT and claims to enable \emph{end-to-end} traffic monitoring~\cite{deepinsight}. However, Barefoot Deep Insight is a proprietary solution without disclosing any technical detail, while our proposal provides a completely open solution to achieve \emph{network-wide} traffic monitoring.

Our major contributions are summarized as follows:

\begin{itemize}

\item We raise ``In-band Network-Wide Telemetry'' and conceive a mechanism-policy-separation framework. Specifically, we propose a source routing-based telemetry mechanism by coupling the INT probe with a source routing label stack for user-specified path monitoring (\S\ref{sec:mechanism}).

\item We develop DFS-based and Euler trail-based path planning algorithms to generate non-overlapped INT paths covering the entire network. The Euler trail-based algorithm can generate theoretically minimum path number to maximally reduce the telemetry overhead (\S\ref{sec:algorithm}).

\item We conduct an exhaustive analysis of algorithm's run-time complexity and derive an upper bound of the algorithm's complexity as $O(k(3E + V -\frac{15}{2}k))$ (\S\ref{subsec:theory}).

\item We implement INT-path prototype with 2995 lines of code, available at our git repository~\cite{git}. The extensive evaluation shows that INT-path is very suitable for today's date center network symmetric topologies (\S\ref{sec:evaluation}).

\end{itemize}

